\documentclass[helvetica]{seminar} 
\input{xy}
\xyoption{all}
\usepackage{graphicx} 
\usepackage{slidesec} 
\usepackage{url}
\usepackage[framemethod=TikZ]{mdframed}
\usepackage{color}

\long\def\symbolfootnote[#1]#2{\begingroup%
\def\thefootnote{\fnsymbol{footnote}}\footnote[#1]{#2}\endgroup}

% to fix problems making landscape seminar pdfs
% Letter...
\pdfpagewidth=11truein
\pdfpageheight=8.5truein
\pdfhorigin=1truein     % default value(?), but doesn't work without
\pdfvorigin=1truein     % default value(?), but doesn't work without
% A4
%\pdfpagewidth=297truemm % your milage may vary....
%\pdfpageheight=210truemm
%\pdfhorigin=1truein     % default value(?), but doesn't work without
%\pdfvorigin=1truein     % default value(?), but doesn't work without


\renewcommand{\familydefault}{\sfdefault}  
 
\input{seminar.bug} 
\input{seminar.bg2} % See the Seminar bugs list 
 
\slideframe{none} 
 
 
\usepackage{fancyhdr} 
 
% Headers and footers personalization using the `fancyhdr' package 
\fancyhf{} % Clear all fields 
\renewcommand{\headrulewidth}{0mm} 
\renewcommand{\footrulewidth}{0.1mm} 
 
\fancyfoot[L]{\tiny IETF 94} 
\fancyfoot[C]{\tiny TLS 1.3 Encrypted SNI}
\fancyfoot[R]{\tiny \theslide} 
 
 
% To center horizontally the headers and footers (see seminar.bug) 
\renewcommand{\headwidth}{\textwidth} 

% To adjust the frame length to the header and footer ones 
\autoslidemarginstrue 
\pagestyle{fancy} 
 

\newcommand{\heading}[1]{% 
  \begin{center} 
    \large\bf 
    #1 
  \end{center} 
  \vspace{.4 in}} 



\begin{document}

\begin{slide}
\begin{center}
\vspace{.5 in}
\LARGE{{\bf}TLS 1.3 Encrypted SNI}\\
\vspace{.2in}
\large{
\begin{tabular}{c}
DKG/EKR
\end{tabular}
}
\end{center}

\end{slide}

\begin{slide}
\heading{DISCLAIMER: THIS IS PROBABLY HALF-BAKED}

We just had this idea yesterday, so it's hand-wavy. No analysis has been done.
\end{slide}


\centerslidesfalse 

\begin{slide}
\heading{Desired security properties}

\begin{enumerate}
\item If you connect to the hidden site, you can learn that someone is covering
   for it and how to connect to the covering site.
\item If you connect to the coverting site, you don't learn that it is covering
   for anyone or who that list is. However, you can *verify* that is covering
   for someone if you suspect that it is.
\item Observation of traffic between the client and gateway/covering site does not allow attackers
   to determine whether the connection is to the the covering site or the
   hidden site.
\end{enumerate}

\end{slide}

\begin{slide}
\heading{Operational Modes}

\begin{itemize}
\item Co-tenanted sites with wildcard certificate
  \begin{itemize}
  \item Client just needs to know it can omit SNI
  \end{itemize}

\item Co-tenanted sites with SAN certificate
  \begin{itemize}
  \item Need encrypted SNI only
  \end{itemize}

\item Gateway server with separate origin server
  \begin{itemize}
  \item The origin server shouldn't see any application-layer traffic
  \item Need something fancier
  \end{itemize}
\end{itemize}

\end{slide}

\begin{slide}
\heading{Gateway Server Flow}

\vspace{-4ex}
\tiny{
$$
\xymatrix@R=.17in@C=1.4in{
\txt{Client} & \txt{Gateway} & \txt{Hidden Server} \\
\ar[r]^{\txt{ClientHello [SNI=innocuous.example.com]}}_{\txt{[EarlyDataIndication, configuration\_id=Y]}} & \\
\ar[r]^{\txt{\emph{EncryptedExtensions [RSNI=hidden.example.com]}}} & \\
\ar[r]^{\txt{\emph{Finished}}} & \\
& \ar[r]^{\txt{ClientHello [SNI=innocuous.example.com]}}_{\txt{[configuration\_id=Y]}} & \\
& & \ar[l]_{\txt{ServerHello [EarlyDataIndication]}} \\
& & \ar[l]_{\txt{\emph{Certificate (hidden)}}} \\
& & \ar[l]_{\txt{...}} \\
& & \ar[l]_{\txt{\emph{Finished}}} \\
\ar[r]^{\txt{\emph{Certificate*, CertificateVerify*}}}_{\txt{\emph{Finished}}} & \\
}
$$
}
\end{slide}

\begin{slide}
\heading{Intuition}

\begin{itemize}
\item The client \emph{knows} that encrypted SNI is in use
  \begin{itemize}
  \item 0-RTT data goes to the gateway \emph{not} to the hidden server
  \item Can't send any application data in 0-RTT
  \end{itemize}

\item So what \emph{certificate} is used to generate keys for 0-RTT?
  \begin{itemize}
  \item Shouldn't be hidden server's certificate (would have to iterate)
  \item So, the gateway's certificate
    \begin{itemize}
    \item This makes sense since we're encrypting to the gateway
    \end{itemize}
  \item TLS doesn't require that these certs be the same
  \end{itemize}

\item Yes, this is a bit weird
\end{itemize}
\end{slide}


\begin{slide}
\heading{How does the client learn about this?}

\begin{itemize}
\item Client needs to know triplet [ServerConfiguration (DH\_s), CSNI, GCERT]
\item Traffic to hidden servers \emph{must} use the same configuration id as traffic to other servers fronted by gateway
\item Client's first connection to hidden server isn't protected.
\end{itemize}
\end{slide}

\begin{slide}
\heading{Possible options}

\vspace{-5ex}
\begin{enumerate}
\item Hidden server sends unsolicited extension with CSNI and GCERT
\item Hidden server sends CSNI and GCERT in ServerConfiguration but in some other part of it that's not hashed into the keys.
\item Hidden server sends a ServerConfiguration with CSNI and GCERT but with same configuration\_id as the ordinary gateway ServerConfiguration. Requires gateway server to do trial decryption.
\item Hidden server delivers the triplet in a non-TLS message (e.g., HTTP header)
\item Hidden server just delivers gateway's domain name somehow and then the
  client connects to the gateway server to get ServerConfiguration.
\end{enumerate}
\end{slide}

\begin{slide}
\vspace{1in}
\heading{Good idea, or the best idea?}

\end{slide}

\end{document}  
                